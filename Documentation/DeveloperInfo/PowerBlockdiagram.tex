\documentclass[border=10pt]{standalone}  
\usepackage{tikz}
\usepackage[european,siunitx,rotatelabels]{circuitikzgit}
\usetikzlibrary{arrows,calc,positioning}

\newcommand{\divider}[1] 
{  % #1 = name 
\begin{scope}[transform shape]
\draw[thick] (#1)node[](a){} +(-12pt,-12pt) rectangle +(12pt,12pt);
\draw (a)+(12pt, 12pt)--+(-12pt,-12pt);
\draw (a)+(-5pt,6pt) node[](){\footnotesize -45};
\draw (a)+(4pt,-7pt) node[](){\footnotesize +45};
\end{scope}
}


\newlength{\ResUp} \newlength{\ResDown}
\newlength{\ResLeft} \newlength{\ResRight}
\newlength{\ResRadius} \newlength{\ResMiddle}

\makeatletter
%%%% new anchors: "out A" and "out B"
\expandafter\def\csname pgf@anchor@rectangle@out A\endcsname{% output A: below .east
    \pgf@process{\southwest}%
    \pgf@ya=0.25\pgf@y%
    \pgf@process{\northeast}%
    \pgf@y=0.75\pgf@y%
    \advance\pgf@y by\pgf@ya    }% end of out A
\expandafter\def\csname pgf@anchor@rectangle@out B\endcsname{% output B: above .east
    \pgf@process{\southwest}%
    \pgf@ya=0.75\pgf@y%
    \pgf@process{\northeast}%
    \pgf@y=0.25\pgf@y%
    \advance\pgf@y by\pgf@ya    } % end of out B
\expandafter\def\csname pgf@anchor@rectangle@DUT\endcsname{% output DUT: above .east
    \pgf@process{\southwest}%
    \pgf@ya=0.1\pgf@y%
    \pgf@process{\northeast}%
    \pgf@y=0.9\pgf@y%
    \advance\pgf@y by\pgf@ya    } % end of DUT
\expandafter\def\csname pgf@anchor@rectangle@ret p\endcsname{% output ret p: above .east
    \pgf@process{\northeast}%
    \pgf@ya=0.9\pgf@y%
    \pgf@process{\southwest}%
    \pgf@y=0.1\pgf@y%
    \advance\pgf@y by\pgf@ya    } % end of ret p
\expandafter\def\csname pgf@anchor@rectangle@ret n\endcsname{% output ret n: above .east
    \pgf@process{\northeast}%
    \pgf@ya=0.7\pgf@y%
    \pgf@process{\southwest}%
    \pgf@y=0.3\pgf@y%
    \advance\pgf@y by\pgf@ya    } % end of ret n
\expandafter\def\csname pgf@anchor@rectangle@ret c\endcsname{% output ret c: above .east
    \pgf@process{\northeast}%
    \pgf@ya=0.8\pgf@y%
    \pgf@process{\southwest}%
    \pgf@y=0.2\pgf@y%
    \advance\pgf@y by\pgf@ya    } % end of ret c


\def\TikzBipolePath#1#2{\pgf@circ@bipole@path{#1}{#2}}
\def\CircDirection{\pgf@circ@direction}
%\pgf@circ@Rlen = \pgfkeysvalueof{/tikz/circuitikz/bipoles/length}
\let\ResLen=\pgf@circ@Rlen
\makeatother

\newcommand{\Compass}% define anchors for compass points
{\anchor{north east}{\northeast}
\anchor{south west}{\southwest}
\anchor{north}{\pgfextracty{\ResUp}{\northeast}\pgfpoint{0cm}{\ResUp}}
\anchor{north west}{\pgfextracty{\ResUp}{\northeast}\pgfextractx{\ResLeft}{\southwest}\pgfpoint{\ResLeft}{\ResUp}}
\anchor{west}{\pgfextractx{\ResLeft}{\southwest}\pgfpoint{\ResLeft}{0cm}}
\anchor{south}{\pgfextracty{\ResDown}{\southwest}\pgfpoint{0cm}{\ResDown}}
\anchor{south east}{\pgfextracty{\ResDown}{\southwest}\pgfextractx{\ResRight}{\northeast}\pgfpoint{\ResRight}{\ResDown}}
\anchor{east}{\pgfextractx{\ResRight}{\northeast}\pgfpoint{\ResRight}{0cm}}}

% ***************************** ADC *********************************
% extra anchors out,in1,in2,vref

\ctikzset{bipoles/ADC/width/.initial=1}
\ctikzset{bipoles/ADC/height/.initial=.5}
\ctikzset{bipoles/ADC/middle/.initial=-.25}
\ctikzset{bipoles/ADC/part/.initial={}}

\def\drawADC{% used by both bipole and node
  \ResRight=\ctikzvalof{bipoles/ADC/width}\ResLen
  \ResRight=0.5\ResRight
  \ResLeft=-\ResRight
  \ResUp=\ctikzvalof{bipoles/ADC/height}\ResLen
  \ResUp=0.5\ResUp
  \ResDown=-\ResUp
  \ResMiddle=\ctikzvalof{bipoles/ADC/middle}\ResLen
  \pgfpathmoveto{\pgfpoint{\ResLeft}{0pt}}
  \pgfpathlineto{\pgfpoint{\ResMiddle}{\ResUp}}
  \pgfpathlineto{\pgfpoint{\ResRight}{\ResUp}}
  \pgfpathlineto{\pgfpoint{\ResRight}{\ResDown}}
  \pgfpathlineto{\pgfpoint{\ResMiddle}{\ResDown}}
  \pgfpathlineto{\pgfpoint{\ResLeft}{0pt}}
  \pgfpathclose
  \pgfusepath{draw}
  \pgftext{\texttt{\ctikzvalof{bipoles/ADC/part}}}
}

\pgfcircdeclarebipole{}% no extra anchors for bipole version
  {\ctikzvalof{bipoles/ADC/height}}
  {ADC}
  {\ctikzvalof{bipoles/ADC/height}}
  {\ctikzvalof{bipoles/ADC/width}}
  {\pgfsetlinewidth{\ctikzvalof{bipoles/thickness}\pgfstartlinewidth}\drawADC}

\def\ADCpath#1{\TikzBipolePath{ADC}{#1}}
\tikzset{ADC/.style = {\circuitikzbasekey, /tikz/to path=\ADCpath, l_=#1}}

\pgfdeclareshape{dADC}{%
\anchor{center}{\pgfpointorigin}    % within the node, (0,0) is the center

\anchor{text}   % this is used to center the text in the node
    {\pgfpoint{-.5\wd\pgfnodeparttextbox}{-.5\ht\pgfnodeparttextbox}}

\savedmacro{\resize}{   % called automatically
  \ResRight=\ctikzvalof{bipoles/ADC/width}\ResLen
  \ResRight=0.5\ResRight
  \ResLeft=-\ResRight
  \ResUp=\ctikzvalof{bipoles/ADC/height}\ResLen
  \ResUp=0.5\ResUp
  \ResDown=-\ResUp
  \ResMiddle=\ctikzvalof{bipoles/ADC/middle}\ResLen
  \ResRadius=\ResMiddle% location of in1 and in2
  \advance\ResRadius by \ResLeft
  \ResRadius=0.5\ResRadius
}% while these can be used for savedanchors, they will be fogotten by anchors

\savedanchor{\northeast}{\pgfpoint{\ResRight}{\ResUp}}
\savedanchor{\southwest}{\pgfpoint{\ResLeft}{\ResDown}}

\savedanchor\InOne{\pgfpoint{\ResRadius}{0.5\ResUp}}
\savedanchor\InTwo{\pgfpoint{\ResRadius}{0.5\ResDown}}
\savedanchor\Out{\pgfpoint{\ResRight}{0pt}}
\savedanchor\Vref{\pgfpoint{0pt}{\ResDown}}

\Compass% standard anchors

\anchor{in1}{\InOne}
\anchor{in2}{\InTwo}
\anchor{out}{\Out}
\anchor{vref}{\Vref}

\foregroundpath{
  \pgfsetlinewidth{\pgfkeysvalueof{/tikz/circuitikz/bipoles/thickness}\pgflinewidth}
  \drawADC}
}

\begin{document}  

\begin{circuitikz}
% Microcontroller
\node[draw, align=center] (pwrswitch) at (0,0) {Power\\Selector};
\node[anchor=east, below left=1cm and 1cm of pwrswitch] (usb) {USB C};
\draw[-latex] (usb.east)-|(pwrswitch.south);
\node[anchor=east, left=1cm of pwrswitch, align=center] (ext) {External\\Power};
\draw[-latex] (ext.east)--(pwrswitch.west);
\coordinate (5V) at (1.5,0);
\draw(pwrswitch.east) to[short, -*](5V);
\node[above] at (5V) {\SI{5}{\volt}};

\node[draw] (boost) at (3,0) {MIC2253};
\node[anchor=south, font=\footnotesize] at (boost.north){Boost converter};
\draw[latex-](boost.west)--(5V);

\node[draw,minimum height=1cm] (buck) at (3,-3) {LM3370};
\node[anchor=south, font=\footnotesize] at (buck.north){Buck converter};
\draw[latex-](buck.west)-|(5V);
\draw[-latex]([yshift=3mm]buck.east)--++(0.5,0)node[anchor=west](){$\SI{3.3}{\volt}_{digital}$};
\draw[-latex]([yshift=-3mm]buck.east)--++(0.5,0)node[anchor=west](){$\SI{1.8}{\volt}_{FPGA}$};

\coordinate (6V) at (7,0);
\draw(boost.east) to[short, -*](6V);
\node[above] at (6V) {\SI{6}{\volt}};

\node[draw] (ldo1) at (9,0) {LDL212PU};
\node[anchor=south, font=\footnotesize] at (ldo1.north){LDO};
\draw[latex-](ldo1.west)--(6V);

\coordinate (5VPLL) at (11,0);
\draw(ldo1.east) to[short, -*](5VPLL);
\node[above] at (5VPLL) {$\SI{5}{\volt}_{PLLs}$};

\node[draw] (ldo2) at (13,0) {NCP702};
\node[anchor=south, font=\footnotesize] at (ldo2.north){LDO};
\draw[latex-](ldo2.west)--(5VPLL);
\draw[-latex](ldo2.east)--++(0.5,0)node[anchor=west](){$\SI{3.3}{\volt}_{Source~CP}$};

\node[draw] (ldo3) at (13,-1) {AP2114H};
\node[anchor=south, font=\footnotesize] at (ldo3.north){LDO};
\draw[latex-] (ldo3.west) -- (ldo3.west-|5VPLL);
\draw (5VPLL) to [short, -*] (ldo3.west-|5VPLL);
\draw[-latex](ldo3.east)--++(0.5,0)node[anchor=west](){$\SI{3.3}{\volt}_{Source~RF}$};

\node[draw] (ldo4) at (13,-2) {NCP702};
\node[anchor=south, font=\footnotesize] at (ldo4.north){LDO};
\draw[latex-] (ldo4.west) -- (ldo4.west-|5VPLL);
\draw (5VPLL) to [short, -*] (ldo4.west-|5VPLL);
\draw[-latex](ldo4.east)--++(0.5,0)node[anchor=west](){$\SI{3.3}{\volt}_{1.LO~CP}$};

\node[draw] (ldo5) at (13,-3) {AP2114H};
\node[anchor=south, font=\footnotesize] at (ldo5.north){LDO};
\draw[latex-] (ldo5.west) -| (ldo4.west-|5VPLL);
\draw[-latex](ldo5.east)--++(0.5,0)node[anchor=west](){$\SI{3.3}{\volt}_{1.LO~RF}$};

\node[draw] (ldo6) at (9,-4) {LDL212PU};
\node[anchor=south, font=\footnotesize] at (ldo6.north){LDO};
\draw[latex-] (ldo6.west) -- (ldo6.west-|6V);
\draw (6V) to [short, -*] (ldo6.west-|6V);
\coordinate (5VADC1) at (11, -4);
\draw(ldo6.east) to [short, -*](5VADC1);
\draw[-latex](5VADC1)--++(3.3,0)node[anchor=west](){$\SI{5}{\volt}_{Port~1~ADC}$};
\node[draw] (ldo7) at (13,-5) {NCP702};
\node[anchor=south, font=\footnotesize] at (ldo7.north){LDO};
\draw[latex-] (ldo7.west) -| (5VADC1);
\draw[-latex](ldo7.east)--++(0.5,0)node[anchor=west](){$\SI{1.8}{\volt}_{Port~1~ADC}$};

\node[draw] (ldo8) at (9,-6) {LDL212PU};
\node[anchor=south, font=\footnotesize] at (ldo8.north){LDO};
\draw[latex-] (ldo8.west) -- (ldo8.west-|6V);
\draw (6V) to [short, -*] (ldo8.west-|6V);
\coordinate (5VADC2) at (11, -6);
\draw(ldo8.east) to [short, -*](5VADC2);
\draw[-latex](5VADC2)--++(3.3,0)node[anchor=west](){$\SI{5}{\volt}_{Port~2~ADC}$};
\node[draw] (ldo9) at (13,-7) {NCP702};
\node[anchor=south, font=\footnotesize] at (ldo9.north){LDO};
\draw[latex-] (ldo9.west) -| (5VADC2);
\draw[-latex](ldo9.east)--++(0.5,0)node[anchor=west](){$\SI{1.8}{\volt}_{Port~2~ADC}$};

\node[draw] (ldo10) at (9,-8) {LDL212PU};
\node[anchor=south, font=\footnotesize] at (ldo10.north){LDO};
\draw[latex-] (ldo10.west) -| (ldo8.west-|6V);
\coordinate (5VADC3) at (11, -8);
\draw(ldo10.east) to [short, -*](5VADC3);
\draw[-latex](5VADC3)--++(3.3,0)node[anchor=west](){$\SI{5}{\volt}_{Reference~ADC}$};
\node[draw] (ldo11) at (13,-9) {NCP702};
\node[anchor=south, font=\footnotesize] at (ldo11.north){LDO};
\draw[latex-] (ldo11.west) -| (5VADC3);
\draw[-latex](ldo11.east)--++(0.5,0)node[anchor=west](){$\SI{1.8}{\volt}_{Reference~ADC}$};

\end{circuitikz}
\end{document}
